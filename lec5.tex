\documentclass[11pt]{article}
\usepackage{amsmath,amssymb,amsthm}

\DeclareMathOperator*{\E}{\mathbb{E}}
\let\Pr\relax
\DeclareMathOperator*{\Pr}{\mathbb{P}}

\newcommand{\eps}{\varepsilon}
\newcommand{\inprod}[1]{\left\langle #1 \right\rangle}
\newcommand{\R}{\mathbb{R}}

\newcommand{\handout}[5]{
  \noindent
  \begin{center}
  \framebox{
    \vbox{
      \hbox to 5.78in { {\bf CS 224: Advanced Algorithms } \hfill #2 }
      \vspace{4mm}
      \hbox to 5.78in { {\Large \hfill #5  \hfill} }
      \vspace{2mm}
      \hbox to 5.78in { {\em #3 \hfill #4} }
    }
  }
  \end{center}
  \vspace*{4mm}
}

\newcommand{\lecture}[4]{\handout{#1}{#2}{#3}{Scribe: #4}{Lecture #1}}

\newtheorem{theorem}{Theorem}
\newtheorem{corollary}[theorem]{Corollary}
\newtheorem{lemma}[theorem]{Lemma}
\newtheorem{observation}[theorem]{Observation}
\newtheorem{proposition}[theorem]{Proposition}
\newtheorem{definition}[theorem]{Definition}
\newtheorem{claim}[theorem]{Claim}
\newtheorem{fact}[theorem]{Fact}
\newtheorem{assumption}[theorem]{Assumption}

% 1-inch margins, from fullpage.sty by H.Partl, Version 2, Dec. 15, 1988.
\topmargin 0pt
\advance \topmargin by -\headheight
\advance \topmargin by -\headsep
\textheight 8.9in
\oddsidemargin 0pt
\evensidemargin \oddsidemargin
\marginparwidth 0.5in
\textwidth 6.5in

\parindent 0in
\parskip 1.5ex

\begin{document}

\lecture{4 --- September 16, 2014}{Fall 2014}{Prof.\ Jelani Nelson}{Albert Wu}

\section{Cuckoo Hashing}

Let us say we have an array $A$ of size $m = 4n$, two random has functions $g, h$. We try
to insert $x$ into $A[g(x)]$, potentially kicking out item already there and moving it. Note that this might
cascade.

If a sequence of items moves goes on for $\geq C \cdot \lg n$ steps, we give up, pick new g and h,
and rebuilt entire data structure.

\paragraph{\underline{Claim:}} $\mathbb{E}(\text{time to insert x}) \leq O(1)$.

\underline{Proof:} A cuckoo graph has $m$ vertices (one per cell of $A$) and
$n$ edges (since for each $x$, we connect $g(x)$ to $h(x)$).

Consider the path we get from an insertion of $x$. We could get a simple path, a single cycle, or a double
cycle. Let us define the following random variables: $T$, the runtime; $P_k$, the indicator random
variable of a path being at least length $k$; $C_k$, the indicator random variable for single-cycle config
of length $\geq k$; and $D$ the indicator for a random variable for having a 2-cycle config. Note also that the probability of the insertion process taking more than $N = C\log n$ steps implies that one of either $D$, $P_N$, or $C_N$ occurred. Therefore

We know that:
\begin{align}
\nonumber \mathbb{E}T =&\ \mathbb{E} \sum_k P_k + \mathbb{E} \sum_k C_k  +P(\text{go on for more than C log n steps}) \cdot n\cdot \mathbb{E} T\\
&{}\le \mathbb{E} \sum_k P_k + \mathbb{E} \sum_k C_k  +(P(D=1) + \mathbb{E} P_N + \mathbb{E} C_N) \cdot n\cdot \mathbb{E} T \label{eqn:all-terms}
\end{align}

Let us consider $\mathbb{E} P_k$. Fix $x_2, x_3, \ldots, x_{k+1}$. Fix the assignment of the $(k + 1)$
hash values to vertices. The probability we see exactly this path is $\frac{1}{m} \cdot \frac{1}{m^{2k}} \cdot 2^k$. To do this, note that
the number of total possible has values is $m^{k+1}$, the number of ways to choose edges is
$n \cdot (n - 1) \cdots (n - k + 1) \leq n^k$.

Then, by union bound, we know that $\E[P_k] \leq n^k \cdot m^{k+1} \cdot \frac{1}{m} \frac{1}{m^{2k}} \cdot 2^k=\frac{1}{2^k}$.

Now, let us bound $C_k$. For $C_k$, let us define 3 types of edges (the forward edges,
the backward edges, and edges on the subsequent path created by the other function).
One of these must have $k/3$ edges, giving us a similar bound as the path analysis $\E[C_k] \leq \frac{1}{2^{k/3}}$.

For $D$, we want $\Pr(D = 1)$. Let $t$ denote the number of distinct vertices (which will also be the number of distinct edges, not including edges labeled with $x$) in the double cycle graph. 
Let $D_t$ be the indicator random variable for having a tour of this type with $t$ vertices. We know that
\begin{equation}
P(D = 1) = \sum_k P(D_t = 1) \label{eqn:converge}
\end{equation}

Let us look for a particular configuration with $t$ vertices. The probability we see this
config is $\frac{1}{m^2} \cdot \frac{1}{(m^2)^t} \cdot 2^t$ (the extra $1/m^2$ comes from requiring $x$ to hash to its two vertices).
Union bounding over all configurations: we have at most $m^t$ choices of vertices, at most $n^t$
choices of edges, and at most $t^3$ choices for the start of the first cycle, the length of the first cycle, and the start of the second cycle. Thus
$$
P(D_t = 1) \le t^3 \cdot \frac{(2mn)^t}{m^{2t+2}} ,
$$
which is at most $(1/n^2) t^3/2^t$. Thus Eq.\ \eqref{eqn:converge} converges and is $O(1/n^2)$.

Now, in Eq.\ \eqref{eqn:all-terms}, the probability of going on for more than $N$ steps is at most $P(D=1) + \E P_N + \E C_N$. By setting $C$ large enough, this is $O(1/n^2)$, dominated by the $P(D=1)$ term. Rearranging terms thus gives $\mathbb{E} T = O(1)$, as desired.



\section{Last Thing on Hashing}

Let us talk about the ``power of two choices.'' Recall hashing w/ chaining. If we choose a perfect random
hash function, with high probability, the length of the longest list is $O\left(\frac{\lg{n}}{\lg\lg{n}}\right)$.

[Azar, Broder, Karlin, Upfal, SICOMP `99] Pick 2 random hash functions $g, h$. When inserting $x$,
place in the least loaded amongst $A[g(x)]$ and $A[h(x)]$. Now, with high probability, the heaviest
bin has at most $\frac{\ln{\ln{n}}}{\ln{2}} + \Theta(1)$ items.

What about the power of $d$ choices? We only improve by a constant factor, i.e., $\frac{\ln\ln n}{\ln d} + \Theta(1)$
items in heaviest.

[V\"{o}cking JACM `03] Break up bins into $d$ groups each of size $n/d$. When insert item, check random
locations in each group. Put in least loaded, break ties by placing in leftmost. Now, the maximum load is
$\Theta\left(\frac{ln\ln n}{d}\right)$.

To see more, see survey by Mitzenmacher, Richa, Sitaraman.

\underline{Intuition for power of 2 choices:}

Let $B_i$ be the number of bins with load $\geq i$. Let the height of $x$, $H(x)$ be such that $x$ is he
$H(x)$th item inserted into that bin.

Let $Q_x$ be the indicator random variable for event that $H(x) \geq i + 1$. The probability that
$H(x) \geq i + 1$ is at most $\left(\frac{B_i}{n}\right)^2$. So, if everything is as expected,
$B_{i+1} \leq n \cdot \left(\frac{B_i}{n}\right)^2$, i.e., $\left(\frac{B_{i+1}}{n}\right) \leq \left(\frac{B_i}{n}\right)^2$.

Let's say that $\frac{B_{10}}{n} \leq \frac{1}{2}$. Then, $\frac{B_{10 + j}}{n} \leq \frac{1}{2^{2^j}}$. We
are done with $B_{10 + j}{n} < \frac{1}{n}$, which append when $j \geq \lg \lg n$.

\underline{Proof:}

Define $\alpha_6 = \frac{n}{2e}, \alpha_{i+1} = \frac{e \alpha_i^2}{n}$. If $E_i$ is the event that $B_i \leq \alpha_i$,
we will show that who all events $E_i$ occur.

First, $\Pr(E_6) = 1$ because $\frac{n}{2e} > \frac{n}{6}$.

\underline{Claim 1:} $\Pr(B_{i+1} > \alpha_{i+ 1} | E_i) \leq C \cdot \Pr(Bin\left(\left(\frac{\alpha_i}{n}\right)^2\right) > \alpha_{i+1})$
\begin{proof}

\end{proof}

Proof left as an exercise to the reader. Uses Chernoff and union bound.

\section{Next Time}

We will talk about data structures + amortized analysis, heaps (binomial and Fibonacci), and splay
trees.

For heaps, we store $n$ items w/keys (comparable). We can insert($x$), decreaseKey($x, k$),
and deleteMin(). Dijkstra's algorithm uses heaps in its implementation, and its runtime is
$m \cdot \text{insert} + m \cdot \text{decreaseKey} + n \cdot \text{deleteMin}$ if there are $n$
vertices and $m$ edges.


\bibliographystyle{alpha}

\begin{thebibliography}{42}

\bibitem{Azar99}
Yossi~Azar, Andrei Z.~Broder, Anna R.~Karlin, Eli~Upfal.
\newblock Balanced Allocations.
\newblock {\em SIAM J. Comput.}, 29(1):180--200, 1999.

\bibitem{Vocking2003}
Berthold~V\"ocking.
\newblock How asymmetry helps load balancing.
\newblock {\em J. ACM.}, 50(4):568--589, 2003.

\bibitem{Mitzenmacher}
Michael~Mitzenmacher, Andr\'{e}a W.~Richa, Ramesh~Sitaraman.
\newblock Chapter 9: The Power of Two Random Choices:
A Survey Of Techniques And Results.
\newblock {\em Handbook of Randomized Computing}. 2001. Kluwer Academic Publishers.
\newblock

\end{thebibliography}

\end{document}